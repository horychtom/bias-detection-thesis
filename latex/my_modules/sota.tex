\chapter{State of the art}

%______________________________GENDER______________________________________________%
\section{Gender bias detection}\label{gender}
Most of the work done on gender bias aims to study gender bias embedded in models and other methods to measure, clarify, and possibly mitigate it.

There is clear evidence that current language models possess implicit gender bias. Whether it means, in terms of learned, biased embeddings \cite{bolukbasi2016man} or simply underrepresentation of a particular gender in the data \cite{sun-peng-2021-men}. 

Yet, my work aspires to classify news texts; therefore, I examined the possibilities of gender classification in text.

I closely followed the approach of Dinan et al. \cite{dinan2020multi}. They define three dimensions of gender bias: bias when speaking \textit{ABOUT} someone, \textit{TO} someone or \textit{AS} someone. Target classes are \{masculine,feminine,neutral\}. 

The word \textbf{bias} here simply means an aspect of the statement that implies the gender of a particular person along the dimensions. To make this definition more clear, for example, the authors further propose that an unbiased sentence would be a sentence in which a machine learning model would not be able to classify a gender, because there would basically be no difference between the classes. Yet, in a real world scenario, sentences \textbf{are} influenced by gender, and therefore such classification is possible.

To measure this kind of bias over all three dimensions, a large-scale dataset \textbf{md\_gender}\footnote{\url{https://huggingface.co/datasets/md_gender_bias}} has been collected. Authors train a transformer model using \Gls{mtl}, to capture all three dimensions. However, only \textit{ABOUT} dimension and very small fraction of \textit{AS} dimension are publicly available, so I only focused on the \textit{ABOUT} dimension.

\begin{itemize}
\item \textbf{md\_gender} - is a collection of automatically labeled large-scale data gathered from various sources around the internet, where gender annotation of a particular dimension is provided (eg., gender information of a user in an internet discussion). It also includes a small gold-labeled dataset for evaluation with 785 data points for the \textit{ABOUT} dimension.
\end{itemize}

To transfer the results of the paper mentioned above to the Czech environment, I sampled 150k sentences from across all datasets with an \textit{ABOUT} dimension label and translated them via \textbf{DeepL} machine translator (more on machine translation in section \ref{DeepL}). Then I managed to train a RoBERTa based model that achieved an F1 score of 80\% on the small gold labeled evaluation dataset. 
Unfortunately, the results are not comparable to the original english experiments, because I took a \textbf{single-task} approach and omitted other dimensions completely. I share the trained model together with translated data on HuggingFace\footnote{\url{https://huggingface.co/}} hub and I also present a demo. An example of the demo can be seen in the Appendix \ref{gender_results}.

The gender classifier, such as this one, can be used to determine what percentage of a particular article in the Czech news environment is about men, women, or is completely genderless. This statistical indicator could help to keep the writing more balanced or give an insight into already published writing.





%___________________________MEDIA_BIAS______________________________________________%
\section{Media bias detection}\label{mediabias}
When it comes to automatic detection of media bias, the standard is to use supervised learning. Most of the prior work done in media bias used hand-crafted features together with traditional\footnote{By traditional I refer to all \Gls{ml} models that are not deep neural networks.} \Gls{ml} algorithms. For example, Hube et al. \cite{hube2018detecting} used a lexicon-based approach with various lexicons (sentiment, bias, subjective, and other linguistic features). Although hand-crafted feature-based approaches offer relatively reasonable explainability, they were outperformed by neural networks and have been replaced by them completely.


ADD: meta-information, source media etc.

The majority of current research focuses on \textbf{sentence level} classification \cite{sinha2021determining,Spinde2021MBIC,lee2021unifying,hube2019neural}, however, the classification can be extrapolated to \textbf{article level}.

Article-level classification is usually more difficult, since it is quite problematic to put the whole article through the neural network. Even though such things as document embeddings exist, bottom-up solutions are usually used. A simple approach would be to classify all sentences and simply count the frequency. Additional high-level features (eg., position of bias) have been studied and have been shown to be effective \cite{chen2020detecting,chen-etal-2020-analyzing}.

As I outlined in the previous section, \gls{mb} can be divided into two classes, where one depends on the outer context and the other does not. This is commonly referred to as \textbf{informational} and \textbf{lexical} bias. There have been efforts to classify informational bias with varying context sizes \cite{van2020context}, although it is a rather unique approach, and so neither I will focus on informational bias in this work.

Various pre-training and fine-tuning strategies have been studied; however, one of the most promising is using an \Gls{mtl} to tackle the problem. Although there are already some results of applying \Gls{mtl} to the detection of \gls{mb} \cite{lee2021unifying,spindeexploiting}, empirical studies suggest that a large number of tasks has to be used to allow \Gls{mtl} truly shine.