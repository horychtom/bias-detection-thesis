\chapter{Conclusion}
In this work, I collected and analyzed the literature and resources to study state-of-the-art media bias detection and also performed a minor experiment focused on gender bias detection. I presented a new Czech parallel dataset derived from Wikipedia with 5 765 sentences and, in addition, nine parallel translated Czech datasets to tackle the detection of media bias in Czech language, one of them large-scale (WNC with 360k sentences).

I trained and tuned the BERT-based FERNET-C5 language model for binary classification and achieved an F1 score of 0.804 on a small test subset of the BABE media bias dataset. I performed experiments on combining different datasets for pre-training the model to push the performance on the validation set. The pre-training on all datasets combined performed the best; however, both hyperparameter tuning and pre-training had generally a very low effect on performance, approximately +0.7\% gain over the baseline. 

Finally, the final classifier has been used to build a publicly available demo and to analyze a sample of articles from the SumeCzech dataset. The results of this study showed a trend in the progression of media bias over time and revealed a positive correlation between the headline bias and the average bias of the article.

\section{Ethical Concerns}
Although the performance of the current classifier is quite appealing, a standalone F1 score might not provide the appropriate evaluation of the ability of the model. Perhaps a human evaluation should also play a role in the process.

Also, the model's decisions are not easily clarifiable. The problem of explainability of the model is especially important when such classifier is brought to real-world applications. 


\section{Future perspective}
As outlined in the introduction (\ref{mb_intro}), according to allsides.com, media bias appears to be a combination of several potentially independent features, such as sentiment, agression, or subjectivity. In this thesis, this hypothesis has also been somewhat supported by the result that pre-training on the subjectivity task had the best influence on the detection of \gls{mb} but a single-task model for subjectivity detection eventually performed worse than others. Therefore, as Spinde et al. \cite{spindeexploiting} suggest, the multi-task approach could be used to improve the classification ability of the current classifier. Therefore, a future collaboration with \gls{mbg} has been established and the \gls{mtl} approach will be thoroughly studied.


Nevertheless, the current Czech classifier relies heavily on the translated datasets. I suggest that, for future improvement, a construction of an original gold-standard Czech dataset is essential.
