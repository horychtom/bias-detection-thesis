\chapter{Czech datasets}
Czech language is a so-called \textbf{low resource language}, which, in the machine learning community, means that, for a particular language, a limited number of datasets of sufficient quality  and size, is available. Thus, the bias detection task in Czech environment is complicated. Despite the relatively sufficient number of datasets in English, there is essentially no Czech one suitable.

 In essence, three options to solve this problem are feasible. The most promising way is to annotate a new gold-standard dataset. However, media bias is a nontrivial, complex, and subtle linguistic feature, hence a lot of effort must be put into annotator training and eventually filtering of implicitly biased annotations.
 
 Another way is to use an automatic approach. \textbf{Allsides}\footnote{\url{https://www.allsides.com/unbiased-balanced-news}} for example, provide annotations on source and article level with expert annotation quality. However, since I focus on a statement level only, using such data leads to oversimplification and results in a very noisy dataset. Regardless, it can still be used for domain-specific pretraining \cite{Spinde2021f}. Unfortunately, there is no Czech site that would provide \textbf{useful} bias information on neither source or article level. Server \Gls{nfnz} \footnote{https://www.nfnz.cz/} provides scoring for different news sources. Yet, only a fraction of their scoring is related to the actual linguistic aspect of the writing. Most of the scoring is based on meta-information such as transparency, proper citation, advertisement, etc.
 
 Nonetheless, automatic creation of a dataset can be done in a clever way like described in section \ref{wiki-npov}. Despite the limitation caused by the size of the particular Wikipedia, this approach is suitable for Czech environment, since Czech Wikipedia has a comparably large editor base\footnote{ \url{https://en.wikipedia.org/wiki/List_of_Wikipedias}} ranking \#26 in a number of edits worldwide. I took this approach and I present a \textbf{new parallel corpus} for bias detection based on Czech Wikipedia \ref{wncs}.
 
 Finally, for low resource languages, it is reasonable to translate English datasets. As one of my contributions to bias detection in Czech news, I reviewed, collected, and translated most of the relevant datasets described in chapter \ref{datasets} using \textbf{DeepL}, and finally processed them into a unified format \ref{processing}. .
 
 
 
 
 
\section{Machine Translation}\label{DeepL}
Since translation of large datasets by human translators would be too costly and from a time perspective practically impossible, automatic machine translation systems are used. In recent years, machine translation, as other fields of \Gls{nlp}, has experienced a massive boost in performance, due to the rise of attention mechanism and complex transformer architectures \ref{att_transformers}.

Modern machine translation models use the \textbf{encoder-decoder} architecture (usually more encoders and decoders stacked on top of each other \ref{att_transformers}), where the encoder part distils (encodes) the information from the input sequence and the decoder part is responsible for decoding this distilled information and mapping it to a sequence in the target language.

For translation of datasets I chose \textbf{DeepL} translator, which is purely\footnote{For example Google combines \Gls{nmt} with statistical approaches, other systems incorporates hardcoded rules, etc.} \Gls{nmt} based system which outperforms other translation systems by a large margin.






\section{Processing}\label{processing}
For convenience, every dataset has been processed into "sentence,label" format, where $label \in \{0,1\}$ stands for \textbf{unbiased} and \textbf{biased}, respectively. Using this simplified data format makes merging and combining several datasets convenient.
    
Moreover, all sentences, which were originally cased were not lower-cased, since I used transformer models that can deal with cased tokens.

\newpage
\section{Translated data}
In the list below, I present all translated datasets in the unified format. I hope this collection will serve as a good starting point for future research of media bias in Czech News.

I will share all listed datasets on HugginFace\footnote{\url{https://huggingface.co/}} hub.

\begin{itemize}
    \item BABE-CS
    \item Basil-CS
    \item WikiBias-CS
    \item CW-hard-CS
    \item MPQA-CS
    \item NFNJ-CS
    \item SUBJ-CS
    \item UA-crisis-CS
    \item WNC-large-CS\footnote{additional \textit{large} is added for distinction between large translated WNC and czech version of WNC}
\end{itemize}

Together, approximately 400k of bias labelled translated sentences were collected.





\section{Czech Wiki Neutrality Corpus}
Finally, I present two novel parallel corpora extracted directly from Czech Wikipedia. To the best of my knowledge, this is the only original Czech dataset related to media bias and subjectivity detection. The only partially relevant dataset is \textbf{SubLex}\cite{11858/00-097C-0000-0022-FF60-B} which is a subjectivity lexicon mainly focused on sentiment. However, lexicon-based approaches are nowadays outperformed by neural models.

I followed two main existing approaches, both of them relying on the extraction of revisions that includes the \{\{NPOV\}\} tag or its variation. The NPOV tag has also its Czech version \Gls{nup}. However, the Czech version is practically not used and so for the extraction, the English variations were used.





\subsection{CWNC-noisy}
I closely followed the \cite{aleksandrova2019multilingual} approach and used their script. Firstly, a file with all pages and its complete edit history is downloaded from the wiki dump\footnote{\url{https://dumps.wikimedia.org/cswiki/}}. I used the \textbf{20220201} version. Then the edits containing one of the NPOV related tags are extracted and then the process of sentence extraction follows. All used tags can be seen in appendix.

This approach yields 15k sentences, however, it uses a rather trivial assumption that when NPOV tag is removed, \textbf{all} removed sentences are biased and all added are unbiased. This annotating strategy led to poor results \ref{experiments}. For this reason, I excluded this dataset from further experiments entirely.





\subsection{CWNC}\label{wncs}
This dataset was created following \cite{pryzant2020automatically} approach. The process is the same as described in section \ref{wiki-npov}. I used \textbf{20220201} snapshot of Wikipedia dump, which was, at the time of dataset collection, the latest snapshot that included all necessary files.
I used the script publicly available on Github\footnote{\url{https://github.com/rpryzant/neutralizing-bias}},with a few slight modifications so the processing fits the Czech language properties:
\begin{enumerate}
    \item Used Regex was extended to exclude czech words that contain "pov" inside eg. \underline{pov}stání, \underline{pov}lak etc.\footnote{Regular expression used to match npov related comments: }
    \item All cases has been preserved, since bert like models can handle cased language.
    \item Czech Morphodita tokenizer was used\footnote{\url{https://ufal.mff.cuni.cz/morphodita/users-manual}}
\end{enumerate}
The final dataset consists of:
\begin{itemize}
    \item 3k of \textit{before} and \textit{after} sentence pairs
    \item 1.7k subset where only one word has been changed
    \item 7.5 sentences, where the change was rejected or reversed, implying neutrality of the original sentence
\end{itemize}

\begin{table}
\makebox[\textwidth][c]{
\begin{tabular}[\textwidth]{|l|}
\hline
\cellcolor{biased_clr} Nizozemsko je známé svým \colorbox{biasedw_clr}{pokrokovým} liberálním postojem vůči psychoaktivním drogám.\\
\cellcolor{unbiased_clr} Nizozemsko je známé svým liberálním postojem vůči psychoaktivním drogám.\\
\hline
\hline
 \cellcolor{biased_clr} Mezi jeho nejznámější a zvlášť populární je jeho hudba ke hrám a filmům, která téměř zlidověla.\\
 \cellcolor{unbiased_clr} Mezi jeho nejznámější a zvlášť populární je jeho hudba k divadelním hrám a filmům, která \\
 \cellcolor{unbiased_clr} v některých případech téměř zlidověla.\\
\hline
\end{tabular}
}
\caption{Example of pairs of biased sentences and their rewritten neutral form}
\label{table:cwnc_example}
\end{table}


In total, 5766 sentences. The neutral corpus,which contains only neutral sentences, is saved for a potential need of oversampling. Two examples of CWNC sentence pairs can be seen in \ref{table:cwnc_example}





\section{Not translated}
Due to a big size of some data, I was not able to translate more than one large-scale dataset. For this reason NewsB dataset has not been translated, since it is one of the few datasets that doesn't focus on the very same task.